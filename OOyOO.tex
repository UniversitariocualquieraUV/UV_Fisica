\documentclass[10pt, twoside]{article}
\usepackage[utf8]{inputenc}

\usepackage{graphicx, amssymb, amsmath, geometry, caption, float, enumitem, fancyhdr, biblatex, multicol, subfigure, comment, hyperref, appendix, tasks, xparse, color, tabularx, booktabs, multirow, xfrac, cancel, ragged2e, fixdif, siunitx, soul, wrapfig, iftex, mathtools, etoolbox, pgfplots, etoc, bm, fontsize, bbm}

\usepackage[skip=10pt]{parskip}

%\usepackage[position=below,belowskip=0pt plus 0pt minus 0pt,aboveskip=0pt]{caption} %esto para nuevos documentos irá bien

%\addtolength\belowcaptionskip{-20pt}

\usepackage[symbol, perpage]{footmisc}
\usepackage[spanish.es]{babel}
\usepackage[dvipsnames]{xcolor}

\usepackage{STY/EPN, STY/slashbox}

%=================== Formato de los entornos ===================%

\renewcommand{\labelenumii}{\arabic{enumi}.\arabic{enumii}}
\renewcommand{\labelenumiii}{\arabic{enumi}.\arabic{enumii}.\arabic{enumiii}}

\renewcommand{\contentsname}{Índice}

\renewcommand{\figurename}{\small Fig}
\counterwithin*{figure}{subsubsection}
\renewcommand{\thefigure}{\small \arabic{section}.\arabic{subsection}.\arabic{subsubsection}.\arabic{figure}}
\captionsetup[figure]{name=Fig}
%\captionsetup[figure]{margin={.1\textwidth,.1\textwidth}} %activar para nuevos documentos

\DeclareCaptionFormat{custom}
{
	\small#1#2 #3
}

\captionsetup{format=custom}

\counterwithin*{table}{section}
\captionsetup[table]{name=Tabla}
\renewcommand{\thetable}{\arabic{section}.\arabic{table}}

\numberwithin{equation}{subsection}

\title{}

\newcommand{\sectiontoc}{%
	\begingroup
	\parbox[b]{0.96\textwidth}{
		\etocsettocstyle{\hrulefill}{\vspace{-0.5em} \hrulefill}%
		\etocsetstyle{subsection}
		{\begin{enumerate}}
			{\normalsize\bfseries\rmfamily\item}
			{\etocname{}\dotfill\etocpage}
			{\end{enumerate}}
		\etocsetstyle{subsubsection}
		{\begin{enumerate}}
			{\normalsize\bfseries\rmfamily\item\normalfont}
			{\etocname{}\dotfill\etocpage}
			{\end{enumerate}}
		\localtableofcontents
	}
	\endgroup
}

%=================== Añade Otro nivel de sección ===================%

\usepackage{titlesec}
\titleformat{\paragraph}
{\normalfont\normalsize\bfseries}{\theparagraph}{1em}{}
\titlespacing*{\paragraph}
{0pt}{3.25ex plus 1ex minus .2ex}{1.5ex plus .2ex}

\makeatletter

\renewcommand\paragraph{\@startsection{paragraph}{4}{\z@}%
	{-2.5ex\@plus -1ex \@minus -.25ex}%
	{1.25ex \@plus .25ex}%
	{\normalfont\normalsize\bfseries}}

\RenewDocumentCommand{\eqref}{mo}{\IfNoValueTF{#2}
	{\textcolor{blue}{\textup{\tagform@{\ref{#1}}}}}
	{\textcolor{blue}{\textup{\tagform@{\ref{#1}\ -\ \ref{#2}}}}}}

\makeatother

\makeatletter

\RenewDocumentCommand{\figref}{O{}m}{\textcolor{blue}{\textup{\tagform@{Fig. \ref{#2} #1}}}}

\NewDocumentCommand{\tabref}{O{}m}{\textcolor{blue}{\textup{\tagform@{Tabla \ref{#2} #1}}}}

\makeatother

\setcounter{secnumdepth}{4} % how many sectioning levels to assign numbers to

%=================== Paquetes para DF ===================%

\usepackage{tikz}
\usetikzlibrary{shapes.geometric, arrows}

\tikzstyle{startstop} = [rectangle, rounded corners, 
minimum width=3cm, 
minimum height=1cm,
text centered, 
draw=black, 
fill=red!30]
\tikzstyle{io} = [trapezium, 
trapezium stretches=true, % A later addition
trapezium left angle=70, 
trapezium right angle=110, 
minimum width=3cm, 
minimum height=1cm, text centered, 
draw=black, fill=blue!30]
\tikzstyle{process} = [rectangle, 
minimum width=3cm, 
minimum height=1cm, 
text centered, 
text width=3cm, 
draw=black, 
fill=orange!30]
\tikzstyle{decision} = [diamond, 
minimum width=3cm, 
minimum height=1cm, 
text centered, 
draw=black, 
fill=green!30]
\tikzstyle{arrow} = [thick,->,>=stealth]

%======================================%

\asignatura{Oscilaciones y Ondas} %Texto en la parte derecha abajo
\carrera{Grado en física}
\titulo{Apuntes Oscilaciones y Ondas}
\grupo{Grupo BL3}
\alumno{Navarro Bonanad, Rubén}
\profesor{Martín Lozano, Víctor}

%===================  Comandos sin argumentos de entrada  ===================%

\newcommand{\comp}{\mathbb{C}}
\newcommand{\re}{\mathbb{R}}
\newcommand{\Tma}{T$^{\underline{ma}}$ }
\newcommand{\Ham}{\mathcal{H}}
\newcommand{\Z}{\mathbb{Z}}
\newcommand{\N}{\mathbb{N}}
\newcommand{\Q}{\mathbb{Q}}
\newcommand{\Lagr}{\mathcal{L}}
\newcommand{\dbar}{{\mathrm{d}\mkern-7mu\mathchar'26\mkern-2mu}}
\newcommand{\grad}{\vec{\nabla}}
\newcommand{\incorrecto}{\textcolor{red}{$\times$}}
\newcommand{\correcto}{\textcolor{ForestGreen}{\checkmark}}
\newcommand{\vsec}[1]{\textcolor{blue}{(v. \textsection #1)}}
\renewcommand{\secref}[1]{\textcolor{blue}{ \textsection #1}}
\newcommand{\vtema}[1]{\textcolor{blue}{(v. Tema #1)}}


%===================  Comandos con argumentos de entrada  ===================%

\newcommand{\Cancel}[2][black]{{\color{#1}\cancel{\color{black}#2}}}
\newcommand{\CR}{\stackrel{\mathclap{\normalfont\mbox{CR}}}{=}}
\newcommand{\geneq}[2]{\stackrel{\mathclap{\normalfont\mbox{#2}}}{#1}}
\newcommand{\cancelcolor}[2]{\textcolor{#1}{\cancel{\textcolor{black}{#2}}}}
\newcommand{\quot}[1]{\textquotedblleft #1\textquotedblright}
\newcommand{\azul}[1]{\textcolor{blue}{#1}}
\newcommand*\circled[1]{\tikz[baseline=(char.base)]{
		\node[shape=circle,draw,inner sep=2pt] (char) {#1};}}

%===============  Comandos con argumentos de entrada opcionales  ===============%

\NewDocumentCommand{\partder}{omm}{
	\IfNoValueTF{#1}
	{\dfrac{\partial#2}{\partial #3}}
	{\dfrac{\partial^{#1}#2}{\partial #3^{#1}}}
}

\NewDocumentCommand{\spartder}{omm}{
	\IfNoValueTF{#1}
	{\sfrac{\partial#2}{\partial #3}}
	{\sfrac{\partial^{#1}#2}{\partial #3^{#1}}}
}

\NewDocumentCommand{\der}{omm}{
	\IfNoValueTF{#1}
	{\frac{\d#2}{\d #3}}
	{\frac{\d^{#1}#2}{\d #3^{#1}}}
}

\NewDocumentCommand{\pdv}{ommm}{
	\IfNoValueTF{#1}
	{\left(\partder{#2}{#3}\right)_{#4}}
	{\left(\partder[#1]{#2}{#3}\right)_{#4}}
}

\NewDocumentCommand{\spdv}{ommm}{
	\IfNoValueTF{#1}
	{\left(\spartder{#2}{#3}\right)_{#4}}
	{\left(\spartder[#1]{#2}{#3}\right)_{#4}}
}

\NewDocumentCommand{\sech}{O{}}{\ \text{sech}^{#1}\ }
\NewDocumentCommand{\cosech}{O{}}{\ \text{sech}^{#1}\ }

\newcommand{\defs}{\geneq{=}{\text{\tiny def}}}

\newcommand{\repeq}[1]{\tag*{\normalfont(\ref{#1})\makebox[0pt][l]{\textsubscript{r}}}}

\tikzset{
	main node/.style={inner sep=0,outer sep=0},
	label node/.style={inner sep=0,outer ysep=.2em,outer xsep=.4em,font=\scriptsize,overlay},
	strike out/.style={shorten <=-.2em,shorten >=-.5em,overlay}
}
\newcommand{\bcancelto}[3][]{\tikz[baseline=(N.base)]{
		\node[main node](N){$#2$};
		\node[label node,#1, anchor=north west] at (N.south east){$#3$};
		\draw[strike out,-latex,#1]  (N.north west) -- (N.south east);
}}

%===================  Definición de entornos  ===================%

\newcounter{colorequation}
\counterwithin*{colorequation}{subsubsection}

\newenvironment{colorequation}[2][black]{
	\stepcounter{colorequation}
	\begin{equation}
		#2 \tag*{\textcolor{#1}{(\thesubsubsection.\thecolorequation)}}
\end{equation}}

%========= inclusión de nuevas fuentes =========%

\iftutex
\usepackage{fontspec}
\usepackage[T1]{fontenc}
\usepackage{tgchorus}

\defaultfontfeatures{ Scale=MatchLowercase, Ligatures=TeX }
\newfontfamily\qtbrush{QT Brush Stroke}
\else
\newcommand\qtbrush{\usefont{T1}{pbsi}{xl}{n}}
\fi

\DeclareSymbolFont{qtbrush}       {T1}{pbsi}{xl}{n}
\DeclareMathSymbol\qtbrushm{\mathalpha}{qtbrush}{`\m}
\DeclareMathSymbol\qtbrushM{\mathalpha}{qtbrush}{`\M}

%=================== Cambio de fuentes y declaración de símbolos ===================%

\usepackage{hyperref, color}
\hypersetup{
	linktocpage=true,
	colorlinks=true,
	linkcolor=blue,
}

%\DeclareSymbolFont{molarvolume}{OT1}{ptm}{m}{it}
\DeclareSymbolFont{doubleu}{OT1}{ptm}{m}{it}
%\DeclareSymbolFont{why}{OT1}{ptm}{m}{it}
\DeclareSymbolFont{zed}{T1}{fvm}{m}{n}
\DeclareMathSymbol{\zed}{\mathalpha}{zed}{`z}
%\DeclareMathSymbol{v}{\mathalpha}{molarvolume}{`v}
\DeclareMathSymbol{w}{\mathalpha}{doubleu}{`w}
%\DeclareMathSymbol{y}{\mathalpha}{why}{`y}
\DeclareMathOperator\artanh{artanh}

\captionsetup[figure]{margin={.05\textwidth,.05\textwidth}}
\captionsetup[table]{margin={.05\textwidth,.05\textwidth}}



\begin{document}
	
\maketitlepage
	
\newpage
	
\pagestyle{fancy}
\fancyhead{} 
\fancyhead[L]{\footnotesize \nouppercase{\leftmark\hfill}}
\fancyhead[R]{\footnotesize Navarro Bonanad, Rubén}
	
%\fancyhead[EL, OR]{\footnotesize \nouppercase{\hyperlink{section.\thesection}{\leftmark}}}
%\fancyhead[ER, OL]{\footnotesize Navarro Bonanad, Rubén}
	
%Esto hace que el encabezado en páginas pares e impares sea distinto, por si se desea imprimir a doble cara, quedaría mejor
	
\setcounter{tocdepth}{2}    % how many sectioning levels to show in ToC
	
\hypertarget{toc}{\tableofcontents}
	
\newpage
	
	
	
\fancyhead[L]{\footnotesize \nouppercase{\hyperlink{section.\thesection}{\leftmark}\hfill} \\ \nouppercase{\rightmark\hfill}}
\fancyhead[R]{\footnotesize \nouppercase{\hyperlink{toc}{Índice}} \\  Navarro Bonanad, Rubén}
	
%\fancyhead[ER, OL]{\footnotesize \nouppercase{\hyperlink{section.\thesection}{\leftmark}\hfill} \\ \nouppercase{\rightmark}}
	
%nuevamente, encabezado basado en la paridad
	
\justifying
	
\setcounter{tocdepth}{3}


	
\section{Oscilaciones simples}

\sectiontoc

Motivación para estudiar ondas: Puede definirse una onda como una colección de oscilaciones correlacionadas

\subsection{Movimiento armónico simple}

\input{T1/Mvmto_arm_simple}

\subsection{Oscilador armónico en 2-Dimensiones}

\input{T1/dos_dim}

\subsection{Oscilaciones amortiguadas}

\input{T1/amort}

\subsection{Formulación Lagrangiana O.A.}

\input{T1/Lagr}

\newpage



\section{Oscilaciones forzadas}

\sectiontoc

\subsection{Fuerza sinusoidal}

\input{T2/fza_sin}

\subsection{Fuerza periódica}

\input{T2/fza_per}

\subsection{Fuerza no periódica}

\input{T2/fza_no_per}

\newpage



\section{Oscilaciones acopladas}

\sectiontoc

\subsection{Acoplamiento de dos osciladores}

\input{T3/T3.1}

\subsection{Acoplamiento débil entre masas}

\input{T3/acop_deb_mas}

\subsection{Sistema de infinitos osciladores}

\input{T3/inf_osc}

\newpage



\section{Ecuación de ondas}

\sectiontoc

Una onda es una perturbación dinámica que se propaga sobre un medio en equilibrio de una o más magnitudes físicas. Las ondas se producen en el medio sin importar su naturaleza. Estos medios pueden ser diversos, tales como el aire o un metal para el caso del sonido, el vacío en el caso de las OEM o un campo de espines alineados. Otra cualidad importante es que la perturbación es dinámica propagándose en el espacio-tiempo de acuerdo a la ecuación de ondas.

Hay dos comportamientos básicos en las ondas, las \textbf{ondas progresivas} (viajan en el espacio) y las \textbf{ondas estacionarias} (su movimiento se repite en la misma región del espacio). Las ondas progresivas pueden formar sistemas dispersivos y no dispersivos.

\begin{itemize}
	\item \textbf{Sistemas no dispersivos}: Todas las ondas viajan con la misma velocidad independiente de la longitud de onda y su frecuencia. Éstas son las ondas que se han visto hasta ahora en dónde la velocidad de propagación de la onda depende del medio y no de las características de la onda.
	
	\item \textbf{Sistemas dispersivos}: En estos sistemas la velocidad de una onda depende de su longitud de onda. De aquí surge un nuevo concepto que es la \textbf{velocidad de grupo}.
\end{itemize}

\subsection{Ondas estacionarias en una dimensión}

\input{T4/T4.1}

\subsection{Instrumentos de viento}

\input{T4/viento}

\subsection{Ondas progresivas. Reflexión y transmisión}

\input{T4/ond_prog}

\subsection{Energía}

\input{T4/energia}

\subsection{Amortiguamiento}

\input{T4/amor}

\subsubsection{Medios dispersivos}

\input{T4/disp}

\newpage



\section{Ondas progresivas}

\sectiontoc

\subsection{}

\input{}


	

\end{document}