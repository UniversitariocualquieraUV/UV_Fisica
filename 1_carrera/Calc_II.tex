\documentclass{article}
\usepackage{graphicx, amsmath, amsthm, amssymb, tikz, mathtools, cancel, indentfirst, mdframed, enumerate, xcolor, multicol, comment, geometry}
\usepackage{fancyhdr}

\newcommand*\circled[1]{\tikz[baseline=(char.base)]{
            \node[shape=circle,draw,inner sep=2pt] (char) {#1};}}

\newcommand{\comp}{\mathbb{C}}
\newcommand{\re}{\mathbb{R}}
\newcommand{\Tma}{T$^{\underline{ma}}$ }
\newcommand{\Tmas}{T$^{\underline{mas}}$ }
\renewcommand{\contentsname}{\center{Índice}}
\newcommand{\partder}[2]{\frac{\partial #1}{\partial #2}}
\newcommand{\Z}{\mathbb{Z}}
\newcommand{\Cancel}[2][black]{{\color{#1}\cancel{\color{black}#2}}}
\newcommand{\CR}{\stackrel{\mathclap{\normalfont\mbox{CR}}}{=}}
\newcommand{\geneq}[2]{\stackrel{\mathclap{\normalfont\mbox{#2}}}{#1}}

\tikzset{
main node/.style={inner sep=0,outer sep=0},
label node/.style={inner sep=0,outer ysep=.2em,outer xsep=.4em,font=\scriptsize,overlay},
strike out/.style={shorten <=-.2em,shorten >=-.5em,overlay}
}
\newcommand{\bcancelto}[3][]{\tikz[baseline=(N.base)]{
  \node[main node](N){$#2$};
  \node[label node,#1, anchor=north west] at (N.south east){$#3$};
  \draw[strike out,-latex,#1]  (N.north west) -- (N.south east);
}}

\usepackage{hyperref, color}
\hypersetup{
    linktocpage=true,
    colorlinks=false,
    linkcolor=blue,
    }

\title{Cálculo II}
\author{Apuntes de Rubén Navarro Bonanad basados en las transparencias y clases \\ de cálculo II de la UV realizadas por Domingo Martínez García}
\date{}

\begin{document}

\maketitle

\tableofcontents

\pagebreak

\pagestyle{fancy}
\fancyhead{} 
\fancyhead[L]{\nouppercase{\leftmark\hfill} \\ \nouppercase{\rightmark\hfill}}
\fancyhead[R]{Apuntes Cálculo II \\ Rubén Navarro Bonanad}

\section{Complementos de cálculo diferencial en $\re^n$}

\subsection{Derivación de funciones compuestas. Regla de la cadena.}

\input{Tema 1/cadena}

\subsection{Derivadas direccionales y gradiente.}

\input{Tema 1/direc_grad}

\subsection{\Tmas de la función implícita e inversa.}

\input{Tema 1/func_imp_inv}

\pagebreak

\section{Derivadas de orden superior. Extremos}

\subsection{Derivadas de orden superior. Fórmula de Taylor en $\re^n$.}

\input{Tema 2/ord_sup_Taylor}

\subsection{Valores extremos y puntos de silla. Matriz Hessiana.}

\input{Tema 2/ext_Hess}

\subsection{Extremos condicionados. Multiplicadores de Lagrange.}

\input{Tema 2/cond_Lag}

\pagebreak

\section{Integrales múltiples}

\subsection{Integrales dobles sobre un rectángulo.}

\input{Tema 3/dob_rec}

\subsection{Integrales dobles sobre regiones elementales.}

\input{Tema 3/dob_elem}

\subsection{Cambio de variable en la integral doble. Coordenadas polares.}

\input{Tema 3/cambio_doble}

\subsection{Integrales triples.}

\input{Tema 3/triple}

\subsection{Cambio de variable en la integral triple. Coordenadas cilíndricas y esféricas.}

\input{Tema 3/cambio_triple}

\subsection{Aplicaciones de las integrales múltiples.}

\input{Tema 3/aplicaciones}

\pagebreak

\section{Campos vectoriales. Cálculo vectorial}

\subsection{Campos vectoriales.}

\input{Tema 4/camp_vect}

\subsection{Operadores diferenciales.}

\input{Tema 4/op_dif}

\subsection{Coordenadas curvilíneas. Vectores y operadores.}

\input{Tema 4/coord_curv}

\pagebreak

\section{Integrales curvilíneas y de superficie}

\subsection{Integrales de línea.}

\input{Tema 5/linea}

\subsection{Integrales de superficie.}

\input{Tema 5/sup}

\subsection{\Tma de Green en el plano.}

\input{Tema 5/Green}

\subsection{\Tma de Stokes}

\input{Tema 5/Stokes}

\subsection{\Tma de Gauss-Ostrogradski.}

\input{Tema 5/Gauss}

\end{document}
